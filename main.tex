\documentclass{article}
\usepackage{graphicx} % Paveikslėliams įterpti
\usepackage{geometry} % Paraščių keitimui
\geometry{a4paper, margin=1in}
\usepackage{hyperref} % Nuorodoms dokumente
\usepackage{float}    % Paveikslų ir lentelių pozicionavimui

\title{PowerPC ir SuperH palyginimas}
\author{Mateušas Zlosnikas}
\date{2024 m. gruodis}

\begin{document}

\maketitle

\tableofcontents % Turinys

\section{Įvadas}
Šiame dokumente palyginamos dvi procesorių architektūros: \textbf{PowerPC} ir \textbf{SuperH}. Abi architektūros turėjo reikšmingą įtaką skirtingoms kompiuterių sritims – nuo įterptųjų sistemų iki aukštos spartos skaičiavimų.

\section{Elementinė kompiuterių bazė}
\subsection{PowerPC}
PowerPC procesoriai buvo sukurti naudojant \textbf{VLSI} (labai didelės integracijos masto) technologiją ir palaipsniui evoliucionavo iki šiuolaikinių monokristalinių mikroprocesorių. Ankstyvieji modeliai buvo paremti \textbf{RISC} (sumažinto komandų rinkinio kompiuterio) dizainu, siekiant aukšto našumo ir mažo sudėtingumo.

\subsection{SuperH}
SuperH procesoriai taip pat buvo pagaminti naudojant \textbf{VLSI} technologiją. Jų dizainas buvo optimizuotas mažam energijos suvartojimui ir kompaktiškam dydžiui, todėl jie plačiai naudoti įterptosiose sistemose, automobilių pramonėje ir mobiliojoje elektronikoje.

\section{Fizinės įrangos savybės}
\subsection{PowerPC}
\begin{itemize}
    \item \textbf{Dydis:} Ankstyvieji modeliai buvo didesni, tačiau technologinė pažanga leido procesoriams sumažėti.
    \item \textbf{Energijos suvartojimas:} Nuo vidutinio iki didelio (50–100 W).
\end{itemize}

\subsection{SuperH}
\begin{itemize}
    \item \textbf{Dydis:} Kompaktiškas dizainas, užimantis vos kelis kvadratinius milimetrus.
    \item \textbf{Energijos suvartojimas:} Itin mažas – nuo 0,5 W iki 10 W.
\end{itemize}

\section{Architektūrų tipai}
\subsection{PowerPC}
PowerPC procesoriai naudoja \textbf{RISC} architektūrą, kuri yra \textbf{registrinio tipo}. Tai reiškia, kad dauguma operacijų atliekamos tarp registrų, o ne tiesiogiai naudojant atmintį. Toks sprendimas leidžia sumažinti delsą ir padidinti našumą.

\begin{itemize}
    \item \textbf{Architektūros tipas:} Registrinė, RISC.
    \item \textbf{Savybės:}
        \begin{itemize}
            \item Fiksuoto ilgio instrukcijos.
            \item Daugybė bendros paskirties registrų.
        \end{itemize}
\end{itemize}

\subsection{SuperH}
SuperH procesoriai taip pat naudoja \textbf{registrinio tipo RISC} architektūrą. Šis dizainas leidžia procesoriams veikti efektyviai tiek energijos, tiek našumo atžvilgiu, todėl jie ypač tinkami įterptosioms sistemoms.

\begin{itemize}
    \item \textbf{Architektūros tipas:} Registrinė, RISC.
    \item \textbf{Savybės:}
        \begin{itemize}
            \item Mažas ir efektyvus komandų rinkinys.
            \item Optimizuotas energijos taupymui.
        \end{itemize}
\end{itemize}
\section{Adresavimo tipai}
\subsection{PowerPC}
PowerPC procesoriai naudoja \textbf{trijų adresų} architektūrą. Instrukcijos dažniausiai turi du šaltinio operandus ir vieną rezultatą, kuris įrašomas į registrą. Ši konstrukcija leidžia atlikti daugiau operacijų per mažesnį instrukcijų skaičių, taip pagerinant našumą.

\begin{itemize}
    \item \textbf{Adresų skaičius:} Trijų adresų.
    \item \textbf{Pavyzdys:} \( R_d = R_s + R_t \), kur \( R_d \) – rezultato registras, o \( R_s \) ir \( R_t \) – šaltinio registrai.
\end{itemize}

\subsection{SuperH}
SuperH procesoriai naudoja \textbf{dviejų adresų} architektūrą. Instrukcija naudoja vieną šaltinio operandą ir rezultatą, kuris gali būti įrašomas į tą patį registrą arba atmintį. Tai supaprastina procesoriaus instrukcijų dekodavimą, tačiau kartais reikalauja papildomų instrukcijų norint atlikti kompleksines operacijas.

\begin{itemize}
    \item \textbf{Adresų skaičius:} Dviejų adresų.
    \item \textbf{Pavyzdys:} \( R_d = R_d + R_s \), kur \( R_d \) veikia kaip tiek šaltinis, tiek rezultatas.
\end{itemize}

\section{Registrai}
\subsection{PowerPC}
PowerPC architektūra turi tiek \textbf{bendrosios paskirties}, tiek \textbf{specializuotus registrus}. Šie registrai yra svarbūs vykdant RISC tipo instrukcijas ir optimizuojant skaičiavimus.

\begin{itemize}
    \item \textbf{Bendrosios paskirties registrai:} 
        \begin{itemize}
            \item PowerPC turi \textbf{32 bendrosios paskirties registrus} (GPR – General Purpose Registers).
            \item Registrų \textbf{duomenų plotis}: 32 bitai ankstyvuose modeliuose, vėlesniuose modeliuose – 64 bitai.
        \end{itemize}
    \item \textbf{Specializuoti registrai:}
        \begin{itemize}
            \item \textbf{LR (Link Register):} Saugoti grįžimo adresą funkcijų kvietimuose.
            \item \textbf{CTR (Count Register):} Naudojamas ciklams ir skaičiavimams.
            \item \textbf{FPSCR (Floating Point Status and Control Register):} Valdo slankaus kablelio operacijas.
            \item \textbf{Specialūs vektoriniai registrai:} Naudojami multimedijos ir signalų apdorojimui (vėlesniuose modeliuose).
        \end{itemize}
\end{itemize}

\subsection{SuperH}
SuperH architektūra turi mažesnį skaičių registrų, tačiau jie yra efektyviai panaudojami, kadangi SuperH yra skirta įterptosioms sistemoms, kur mažas energijos suvartojimas ir greitis yra itin svarbūs.

\begin{itemize}
    \item \textbf{Bendrosios paskirties registrai:}
        \begin{itemize}
            \item SuperH turi \textbf{16 bendrosios paskirties registrų} (R0–R15).
            \item Registrų \textbf{duomenų plotis}: 32 bitai.
        \end{itemize}
    \item \textbf{Specializuoti registrai:}
        \begin{itemize}
            \item \textbf{PC (Program Counter):} Laiko vykdomos instrukcijos adresą.
            \item \textbf{SR (Status Register):} Saugomi procesoriaus būsenos bitai.
            \item \textbf{SP (Stack Pointer):} Valdo steko operacijas.
            \item \textbf{MAC (Multiply-Accumulate Register):} Naudojamas dauginimo ir sumavimo operacijoms.
        \end{itemize}
\end{itemize}

\section*{Registrų palyginimas}
\begin{itemize}
    \item \textbf{PowerPC:} 32 bendrosios paskirties registrai, duomenų plotis iki 64 bitų, daug specializuotų registrų.
    \item \textbf{SuperH:} 16 bendrosios paskirties registrų, 32 bitų duomenų plotis, mažiau specializuotų registrų.
\end{itemize}

\section{Požymių bitai}
\subsection{PowerPC}
PowerPC architektūroje požymių bitai yra naudojami tam tikriems procesoriaus būsenos signalams valdyti. Jie yra saugomi specializuotame registruose, tokiuose kaip \textbf{Condition Register (CR)} ir \textbf{Floating Point Status and Control Register (FPSCR)}.

\begin{itemize}
    \item \textbf{CR (Condition Register):}
        \begin{itemize}
            \item CR registras turi \textbf{8 laukus} (CR0–CR7), kur kiekvienas laukas yra 4 bitų.
            \item Požymių bitai atspindi instrukcijų rezultatus, tokius kaip:
                \begin{itemize}
                    \item \textbf{EQ (Equal):} Reikšmės lygios.
                    \item \textbf{LT (Less Than):} Reikšmė mažesnė.
                    \item \textbf{GT (Greater Than):} Reikšmė didesnė.
                    \item \textbf{SO (Summary Overflow):} Viršijo aritmetinės operacijos reikšmę.
                \end{itemize}
        \end{itemize}
    \item \textbf{FPSCR (Floating Point Status and Control Register):}
        \begin{itemize}
            \item Naudojamas slankaus kablelio operacijų požymių bitams saugoti.
            \item Valdo tokias būsenas kaip **viršpildymas**, **nulio dalyba** ir **neapibrėžtos reikšmės**.
        \end{itemize}
\end{itemize}

\subsection{SuperH}
SuperH architektūroje požymių bitai taip pat egzistuoja, tačiau jų kiekis ir funkcionalumas yra supaprastintas, atsižvelgiant į architektūros efektyvumą.

\begin{itemize}
    \item \textbf{SR (Status Register):}
        \begin{itemize}
            \item SR registras saugo požymių bitus, kurie nurodo procesoriaus būseną.
            \item Dažniausiai naudojami bitai:
                \begin{itemize}
                    \item \textbf{T (T-bit):} Rezultato tikrinimo bitas (naudojamas sąlyginiams šakoms).
                    \item \textbf{I (Interrupt Mask):} Pertraukčių valdymas.
                    \item \textbf{S (Supervisor Mode):} Indikuoja veikimą administratoriaus režimu.
                \end{itemize}
        \end{itemize}
\end{itemize}

\section*{Požymių bitų palyginimas}
\begin{itemize}
    \item \textbf{PowerPC:} Išsamesnė požymių sistema su **CR** ir **FPSCR** registrais, palaikančiais aritmetinių ir slankaus kablelio operacijų būsenas.
    \item \textbf{SuperH:} Paprastesnė sistema su **SR** registru, kuriame yra pagrindiniai bitai, reikalingi šakoms ir pertrauktims valdyti.
\end{itemize}

\section{Duomenų plotis (mašininis žodis)}
\subsection{PowerPC}
PowerPC architektūros duomenų plotis priklauso nuo procesoriaus kartos:
\begin{itemize}
    \item Ankstyvieji PowerPC procesoriai turėjo \textbf{32 bitų} duomenų plotį.
    \item Vėlesnėse versijose, tokiose kaip PowerPC 64, buvo pereita prie \textbf{64 bitų} duomenų pločio.
\end{itemize}
Mašininis žodis nustato maksimalų duomenų kiekį, kurį procesorius gali apdoroti vienu ciklu, todėl 64 bitų versijos leido gerokai didesnį našumą ir adresavimo erdvę.

\subsection{SuperH}
SuperH architektūra buvo sukurta kaip 32 bitų sistema:
\begin{itemize}
    \item Visi bendrosios paskirties registrai ir mašininiai žodžiai yra \textbf{32 bitų}.
    \item Instrukcijų ilgis: \textbf{16 bitų} fiksuoto ilgio komandos, optimizuotos mažoms atminties sąnaudoms.
\end{itemize}
Šis dizainas leido sumažinti energijos suvartojimą ir padidinti instrukcijų vykdymo greitį, ypač įterptosiose sistemose.

\section{Atminties išdėstymas}
\subsection{PowerPC}
PowerPC architektūra naudoja \textbf{ištisinę adresų erdvę} su galimybe palaikyti virtualią atmintį. 

\begin{itemize}
    \item \textbf{Adreso plotis:} 
        \begin{itemize}
            \item 32 bitai ankstyvuose modeliuose.
            \item 64 bitai vėlesniuose modeliuose (PowerPC 64).
        \end{itemize}
    \item \textbf{Maksimalus atminties kiekis:}
        \begin{itemize}
            \item 32 bitų sistemose – iki \( 4 \, \text{GB} \) (2\textsuperscript{32} adresų).
            \item 64 bitų sistemose – iki \( 16 \, \text{EB} \) (2\textsuperscript{64} adresų).
        \end{itemize}
    \item \textbf{Tipiškas atminties kiekis:}
        \begin{itemize}
            \item Ankstyvose sistemose – nuo \( 64 \, \text{MB} \) iki \( 512 \, \text{MB} \).
            \item Naujesnėse serverių sistemose – nuo kelių GB iki kelių TB.
        \end{itemize}
    \item \textbf{Atminties valdymas:}
        \begin{itemize}
            \item Virtualios atminties valdymas su puslapių peradresavimu (MMU – Memory Management Unit).
            \item Ištisinė erdvė leidžia tiesioginį adresavimą.
        \end{itemize}
\end{itemize}

\subsection{SuperH}
SuperH architektūra naudoja \textbf{ištisinę adresų erdvę} su galimybe taikyti segmentuotą atminties valdymą dėl savo paprasto dizaino.

\begin{itemize}
    \item \textbf{Adreso plotis:} 
        \begin{itemize}
            \item 32 bitai, leidžiantys adresuoti iki \( 4 \, \text{GB} \) atminties.
        \end{itemize}
    \item \textbf{Maksimalus atminties kiekis:}
        \begin{itemize}
            \item Iki \( 4 \, \text{GB} \) dėl 32 bitų adresų pločio.
        \end{itemize}
    \item \textbf{Tipiškas atminties kiekis:}
        \begin{itemize}
            \item Dažniausiai nuo \( 8 \, \text{MB} \) iki \( 64 \, \text{MB} \) įterptosioms sistemoms.
        \end{itemize}
    \item \textbf{Atminties valdymas:}
        \begin{itemize}
            \item Nėra pilnos virtualios atminties valdymo sistemos ankstyvuose modeliuose.
            \item Atminties adresai dažnai tiesiogiai naudojami be sudėtingo segmentavimo.
        \end{itemize}
\end{itemize}

\section*{Atminties išdėstymo palyginimas}
\begin{itemize}
    \item \textbf{PowerPC:} Ištisinė adresų erdvė su virtualios atminties valdymu ir MMU. Adreso plotis 32/64 bitai, maksimalus atminties kiekis iki \( 16 \, \text{EB} \).
    \item \textbf{SuperH:} Ištisinė adresų erdvė, 32 bitų adreso plotis, maksimalus atminties kiekis \( 4 \, \text{GB} \), su ribotu atminties valdymu.
\end{itemize}

\section{Virtualiosios atminties palaikymas}
\subsection{PowerPC}
PowerPC architektūra palaiko virtualią atmintį, kuri yra realizuota naudojant \textbf{puslapiavimą}. Tai atliekama naudojant atminties valdymo įrenginį (\textbf{MMU} – Memory Management Unit), kuris atsakingas už adresų pervedimą iš virtualiųjų į fizinius adresus.

\begin{itemize}
    \item \textbf{Puslapiavimas:}
        \begin{itemize}
            \item Atmintis yra padalinta į fiksuoto dydžio puslapius (pvz., 4 KB dydžio).
            \item Virtualūs adresai susiejami su fiziniais naudojant puslapių lenteles (Page Tables).
        \end{itemize}
    \item \textbf{Segmentavimas:} 
        \begin{itemize}
            \item Nors pagrindinis mechanizmas yra puslapiavimas, kai kuriose implementacijose gali būti derinamas su segmentavimu.
        \end{itemize}
    \item \textbf{MMU funkcijos:}
        \begin{itemize}
            \item Palaiko puslapių apsaugą (Read/Write/Execute leidimai).
            \item Leidžia naudoti virtualią atmintį, kuri gali būti didesnė nei fizinė.
        \end{itemize}
\end{itemize}

\subsection{SuperH}
SuperH architektūroje virtualios atminties palaikymas yra ribotas ir priklauso nuo konkretaus procesoriaus modelio. Ankstyvieji modeliai neturėjo virtualios atminties valdymo, o vėlesnėse versijose buvo pridėtas dalinis palaikymas naudojant \textbf{puslapiavimą}.

\begin{itemize}
    \item \textbf{Puslapiavimas:}
        \begin{itemize}
            \item Naudojamas tik kai kuriuose naujesniuose SuperH procesoriuose su MMU.
            \item Atmintis padalinta į fiksuoto dydžio puslapius (pvz., 1 KB arba 4 KB dydžio).
        \end{itemize}
    \item \textbf{Segmentavimas:}
        \begin{itemize}
            \item Segmentavimas plačiai nebuvo naudojamas SuperH architektūroje dėl paprastesnio dizaino.
        \end{itemize}
    \item \textbf{Be virtualios atminties:}
        \begin{itemize}
            \item Ankstyvieji SuperH procesoriai dirbo tik su fiziniais adresais ir tiesioginiu atminties adresavimu.
        \end{itemize}
\end{itemize}

\section*{Virtualios atminties palyginimas}
\begin{itemize}
    \item \textbf{PowerPC:} Pilnas virtualios atminties palaikymas naudojant puslapiavimą ir MMU, su galimybe derinti su segmentavimu.
    \item \textbf{SuperH:} Ribotas palaikymas – ankstyvieji modeliai nepalaikė virtualios atminties, o vėlesniuose modeliuose įdiegtas puslapiavimas.
\end{itemize}
\section{Komandų sistema (ISA)}
\subsection{PowerPC}
PowerPC architektūra naudoja \textbf{RISC (Reduced Instruction Set Computer)} tipo komandų sistemą. Ji pasižymi mažesniu, bet efektyviu komandų rinkiniu, kuris optimizuoja vykdymo greitį.

\begin{itemize}
    \item \textbf{Komandų skaičius:} Apie 200–300 instrukcijų (priklausomai nuo versijos).
    \item \textbf{Instrukcijų klasės:}
        \begin{itemize}
            \item \textbf{Aritmetinės:} Sudėtis, atimtis, daugyba.
            \item \textbf{Loginės:} AND, OR, XOR, NOT.
            \item \textbf{Perkėlimo:} Duomenų judėjimas tarp registrų ir atminties.
            \item \textbf{Šakų (Branch):} Sąlyginiai ir nesąlyginiai šuoliai.
            \item \textbf{Slankaus kablelio:} Aritmetika su slankiuoju kableliu.
            \item \textbf{Specialios:} Sistemos valdymo instrukcijos.
        \end{itemize}
    \item \textbf{Instrukcijų formatai:}
        \begin{itemize}
            \item Fiksuoto dydžio 32 bitų instrukcijos.
            \item Rūšys: \textbf{R-formatas} (registrų operacijos), \textbf{I-formatas} (immediate reikšmės), \textbf{B-formatas} (šakos).
        \end{itemize}
\end{itemize}

\textbf{Instrukcijų pavyzdžiai:}
\begin{itemize}
    \item \texttt{add r3, r4, r5} – Sudeda \texttt{r4} ir \texttt{r5}, rezultatą įrašo į \texttt{r3}.
    \item \texttt{sub r3, r4, r5} – Atima \texttt{r5} iš \texttt{r4}, rezultatą įrašo į \texttt{r3}.
    \item \texttt{and r3, r4, r5} – Atlieka loginę AND tarp \texttt{r4} ir \texttt{r5}, rezultatą įrašo į \texttt{r3}.
    \item \texttt{or r3, r4, r5} – Atlieka loginę OR tarp \texttt{r4} ir \texttt{r5}.
    \item \texttt{lwz r3, 0(r4)} – Įkrauna žodį iš atminties (adresu \texttt{r4}) į \texttt{r3}.
    \item \texttt{stw r3, 0(r4)} – Saugo žodį iš \texttt{r3} į atmintį (adresu \texttt{r4}).
    \item \texttt{b label} – Besąlyginis šuolis į nurodytą etiketę.
    \item \texttt{bl label} – Šuolis su grįžimo adresu į \texttt{Link Register}.
\end{itemize}

\subsection{SuperH}
SuperH architektūra taip pat naudoja \textbf{RISC} komandų sistemą, bet pasižymi mažesniu instrukcijų rinkiniu, kuris optimizuotas įterptosioms sistemoms.

\begin{itemize}
    \item \textbf{Komandų skaičius:} Apie 50–100 instrukcijų.
    \item \textbf{Instrukcijų klasės:}
        \begin{itemize}
            \item \textbf{Aritmetinės:} Sudėtis, atimtis, daugyba.
            \item \textbf{Loginės:} AND, OR, XOR.
            \item \textbf{Perkėlimo:} Duomenų perkėlimas tarp registrų ir atminties.
            \item \textbf{Šakų (Branch):} Sąlyginiai ir nesąlyginiai šuoliai.
        \end{itemize}
    \item \textbf{Instrukcijų formatai:}
        \begin{itemize}
            \item Fiksuoto dydžio \textbf{16 bitų} instrukcijos.
        \end{itemize}
\end{itemize}

\textbf{Instrukcijų pavyzdžiai:}
\begin{itemize}
    \item \texttt{add r0, r1} – Sudeda \texttt{r0} ir \texttt{r1}, rezultatą įrašo į \texttt{r0}.
    \item \texttt{sub r0, r1} – Atima \texttt{r1} iš \texttt{r0}, rezultatą įrašo į \texttt{r0}.
    \item \texttt{mov r0, r1} – Perkelia reikšmę iš \texttt{r1} į \texttt{r0}.
    \item \texttt{and r0, r1} – Atlieka loginę AND tarp \texttt{r0} ir \texttt{r1}.
    \item \texttt{or r0, r1} – Atlieka loginę OR tarp \texttt{r0} ir \texttt{r1}.
    \item \texttt{tst r0, r1} – Patikrina loginius bitus tarp \texttt{r0} ir \texttt{r1}.
    \item \texttt{bra label} – Besąlyginis šuolis į nurodytą etiketę.
    \item \texttt{braf r0} – Šuolis su adresu iš \texttt{r0}.
\end{itemize}

\section*{Komandų palyginimas}
\begin{itemize}
    \item \textbf{Panašumai:} Abi architektūros palaiko RISC dizainą ir turi panašias instrukcijų klases (aritmetinės, loginės, šakų ir perkėlimo komandos).
    \item \textbf{Skirtumai:}
        \begin{itemize}
            \item PowerPC palaiko daugiau komandų (200–300) ir turi 32 bitų instrukcijas.
            \item SuperH turi mažesnį instrukcijų rinkinį (50–100) su 16 bitų instrukcijomis, skirtomis mažai atminties sąnaudai.
        \end{itemize}
\end{itemize}

\section{Adresavimo būdai}
\subsection{PowerPC}
PowerPC architektūra palaiko įvairius adresavimo būdus, kurie užtikrina lankstumą ir efektyvumą vykdant komandas. Pagrindiniai būdai:

\begin{itemize}
    \item \textbf{Tiesioginis adresavimas (Immediate):}
        \begin{itemize}
            \item Naudojama konstanta kaip operandas.
            \item Pavyzdys: \texttt{addi r3, r4, 10} – Prideda reikšmę 10 prie \texttt{r4}, rezultatą įrašo į \texttt{r3}.
        \end{itemize}
    \item \textbf{Registrinis adresavimas:}
        \begin{itemize}
            \item Duomenys paimami tiesiai iš registro.
            \item Pavyzdys: \texttt{add r3, r4, r5}.
        \end{itemize}
    \item \textbf{Bazinis adresavimas (Base/Displacement):}
        \begin{itemize}
            \item Atminties adresas apskaičiuojamas pridedant poslinkį prie bazinio registro.
            \item Pavyzdys: \texttt{lwz r3, 4(r4)} – Įkrauna žodį iš atminties adresu \texttt{r4 + 4}.
        \end{itemize}
    \item \textbf{Indeksuotas adresavimas:}
        \begin{itemize}
            \item Bazinis adresas ir indeksas saugomi registruose.
            \item Pavyzdys: \texttt{lwzx r3, r4, r5} – Apskaičiuoja adresą \texttt{r4 + r5} ir įkrauna duomenis į \texttt{r3}.
        \end{itemize}
    \item \textbf{Šakų adresavimas:}
        \begin{itemize}
            \item Adresai naudojami šakų ir šuolių instrukcijose.
            \item Pavyzdys: \texttt{b label}.
        \end{itemize}
\end{itemize}

\subsection{SuperH}
SuperH architektūra palaiko mažesnį adresavimo režimų rinkinį dėl RISC architektūros ir mažo instrukcijų ilgio (16 bitų). Pagrindiniai būdai:

\begin{itemize}
    \item \textbf{Tiesioginis adresavimas (Immediate):}
        \begin{itemize}
            \item Konstanta nurodoma kaip operandas.
            \item Pavyzdys: \texttt{mov \#10, r0} – Įrašo reikšmę 10 į registrą \texttt{r0}.
        \end{itemize}
    \item \textbf{Registrinis adresavimas:}
        \begin{itemize}
            \item Duomenys paimami iš registro.
            \item Pavyzdys: \texttt{add r0, r1} – Sudeda \texttt{r0} ir \texttt{r1}, rezultatą įrašo į \texttt{r0}.
        \end{itemize}
    \item \textbf{Netiesioginis adresavimas per registrą:}
        \begin{itemize}
            \item Atminties adresas nurodomas registre.
            \item Pavyzdys: \texttt{mov.l @r1, r0} – Įkrauna ilgą žodį iš adreso, esančio registre \texttt{r1}, į \texttt{r0}.
        \end{itemize}
    \item \textbf{Netiesioginis adresavimas su poslinkiu:}
        \begin{itemize}
            \item Adresas apskaičiuojamas pridedant poslinkį prie registro reikšmės.
            \item Pavyzdys: \texttt{mov.l 4(r1), r0} – Įkrauna duomenis iš adreso \texttt{r1 + 4}.
        \end{itemize}
    \item \textbf{Šakų adresavimas:}
        \begin{itemize}
            \item Naudojamas šakoms ir šuoliams.
            \item Pavyzdys: \texttt{bra label}.
        \end{itemize}
\end{itemize}

\section*{Adresavimo režimų palyginimas}
\begin{itemize}
    \item \textbf{Panašumai:}
        \begin{itemize}
            \item Abi architektūros palaiko tiesioginį, registrinį ir šakų adresavimą.
            \item Abi palaiko netiesioginį adresavimą su poslinkiu.
        \end{itemize}
    \item \textbf{Skirtumai:}
        \begin{itemize}
            \item PowerPC palaiko \textbf{indeksuotą adresavimą}, kuris leidžia naudoti du registrus adreso skaičiavimui.
            \item SuperH palaiko mažiau adresavimo režimų dėl 16 bitų instrukcijų ilgio ir paprastesnio dizaino.
        \end{itemize}
\end{itemize}

\section{Įvesties/išvesties (I/O) galimybės}
\subsection{PowerPC}
PowerPC architektūra pasižymėjo plačiomis ir lankstiomis I/O galimybėmis, kurios pritaikytos įvairiems taikymo scenarijams – nuo darbalaukio kompiuterių iki serverių ir įterptųjų sistemų.

\begin{itemize}
    \item \textbf{Atminties žemėlapio I/O (Memory-Mapped I/O):}
        \begin{itemize}
            \item I/O įrenginiai adresuojami kaip atminties erdvės dalis.
            \item Leidžia efektyvų duomenų apsikeitimą tarp procesoriaus ir I/O įrenginių.
        \end{itemize}
    \item \textbf{Specialūs I/O registrai:}
        \begin{itemize}
            \item PowerPC procesoriai turi specializuotus registrus I/O valdymui.
            \item Tai leidžia greitai valdyti periferinius įrenginius.
        \end{itemize}
    \item \textbf{Išplėtimo sąsajos:}
        \begin{itemize}
            \item Palaikomos tokios sąsajos kaip PCI, PCI-X, AGP ir vėlesnėse versijose PCIe.
            \item Plačiai naudota serveriuose ir konsolėse, tokiose kaip Xbox 360.
        \end{itemize}
    \item \textbf{Tiesioginė atminties prieiga (DMA):}
        \begin{itemize}
            \item Leidžia I/O įrenginiams tiesiogiai keistis duomenimis su atmintimi, apeinant procesorių.
            \item Gerina našumą ir mažina CPU apkrovą.
        \end{itemize}
\end{itemize}

\subsection{SuperH}
SuperH architektūra buvo sukurta įterptosioms sistemoms, todėl jos I/O galimybės buvo supaprastintos ir pritaikytos mažos galios periferinių įrenginių valdymui.

\begin{itemize}
    \item \textbf{Atminties žemėlapio I/O (Memory-Mapped I/O):}
        \begin{itemize}
            \item I/O įrenginiai integruojami į procesoriaus adresų erdvę.
            \item Tai leidžia paprastą ir tiesioginį periferinių įrenginių valdymą.
        \end{itemize}
    \item \textbf{Tiesioginė atminties prieiga (DMA):}
        \begin{itemize}
            \item SuperH palaiko DMA valdymą mažiems duomenų srautams perduoti tiesiogiai tarp įrenginių ir atminties.
        \end{itemize}
    \item \textbf{I/O prievadai:}
        \begin{itemize}
            \item I/O prievadai naudojami tiesioginiam periferinių įrenginių valdymui (pvz., GPIO prievadai, UART sąsaja).
            \item Tai itin svarbu įterptosiose sistemose, kur reikalingas tiesioginis valdymas su mažomis energijos sąnaudomis.
        \end{itemize}
    \item \textbf{Serijinės sąsajos:}
        \begin{itemize}
            \item Plačiai naudojamos serijinės komunikacijos sąsajos, tokios kaip UART, SPI ir I\textsuperscript{2}C.
            \item Tai efektyvūs sprendimai mažos spartos įrenginių prijungimui.
        \end{itemize}
\end{itemize}

\section*{I/O galimybių palyginimas}
\begin{itemize}
    \item \textbf{PowerPC:}
        \begin{itemize}
            \item Palaiko sudėtingesnes I/O sąsajas, tokias kaip PCI ir PCIe.
            \item Turi DMA valdymą dideliems duomenų srautams.
            \item Naudojamas atminties žemėlapio I/O ir specializuoti I/O registrai.
        \end{itemize}
    \item \textbf{SuperH:}
        \begin{itemize}
            \item Supaprastintas I/O modelis, skirtas mažos spartos periferiniams įrenginiams.
            \item Palaiko DMA mažiems duomenų srautams ir serijines sąsajas, tokias kaip UART, SPI ir I\textsuperscript{2}C.
            \item Efektyviai naudoja atminties žemėlapio I/O.
        \end{itemize}
\end{itemize}

\section{Pertraukimų mechanizmai}
\subsection{PowerPC}
PowerPC architektūra palaiko išplėstinį pertraukčių valdymą, kuris naudojamas aukštos spartos ir sudėtingose sistemose. Pagrindiniai pertraukimų mechanizmai:

\begin{itemize}
    \item \textbf{Tipai:}
        \begin{itemize}
            \item \textbf{Išorinės pertrauktys:} Generuojamos iš periferinių įrenginių ar kitų išorinių šaltinių.
            \item \textbf{Vidinės pertrauktys:} Sukeliamos programinės klaidos (pvz., dalyba iš nulio) ar instrukcijų vykdymo išimtys.
            \item \textbf{Laiko pertrauktys:} Sukeliamos laikmačio arba skaitiklio veikimo.
        \end{itemize}
    \item \textbf{Pertraukimų valdymas:}
        \begin{itemize}
            \item Naudojamas specialus \textbf{MSR (Machine State Register)}, kuris valdo pertraukimų įjungimą/išjungimą ir procesoriaus būseną.
            \item \textbf{IVOR (Interrupt Vector Offset Registers):} Nurodo pertraukčių tvarkymo vektorių adresus.
        \end{itemize}
    \item \textbf{Prioritetai:}
        \begin{itemize}
            \item PowerPC palaiko pertraukimų prioriteto lygmenis, leidžiančius valdyti svarbesnes pertrauktis.
        \end{itemize}
\end{itemize}

\subsection{SuperH}
SuperH architektūra taip pat palaiko pertraukimų mechanizmą, tačiau jis yra supaprastintas dėl įterptųjų sistemų poreikių.

\begin{itemize}
    \item \textbf{Tipai:}
        \begin{itemize}
            \item \textbf{Išorinės pertrauktys:} Sukeliamos iš periferinių įrenginių ar išorinių signalų.
            \item \textbf{Vidinės pertrauktys:} Sukeliamos programinių klaidų (pvz., neteisingo adreso prieiga).
            \item \textbf{Laiko pertrauktys:} Generuojamos laikmačio įvykių.
        \end{itemize}
    \item \textbf{Pertraukimų valdymas:}
        \begin{itemize}
            \item \textbf{SR (Status Register)}: Yra specialus \textbf{I (Interrupt Mask)} bitas, kuris leidžia įjungti arba išjungti pertraukimus.
            \item Naudojamas \textbf{vektorinis pertraukimų mechanizmas}, kuriame kiekvienai pertraukčiai priskirtas vektoriaus adresas.
        \end{itemize}
    \item \textbf{Prioritetai:}
        \begin{itemize}
            \item Palaikomi fiksuoti pertraukimų prioritetai su ribotu lygių skaičiumi.
        \end{itemize}
\end{itemize}

\section*{Pertraukimų mechanizmų palyginimas}
\begin{itemize}
    \item \textbf{Panašumai:}
        \begin{itemize}
            \item Abi architektūros palaiko išorines, vidines ir laiko pertrauktis.
            \item Abi architektūros naudoja vektorinius pertraukimų mechanizmus.
            \item Pertraukimai gali būti įjungiami arba išjungiami naudojant specialius registrus (\textbf{MSR} PowerPC ir \textbf{SR} SuperH).
        \end{itemize}
    \item \textbf{Skirtumai:}
        \begin{itemize}
            \item \textbf{PowerPC:} 
                \begin{itemize}
                    \item Palaiko sudėtingesnį pertraukimų prioritetų valdymą.
                    \item Naudoja specializuotus registrus, tokius kaip \textbf{IVOR}.
                \end{itemize}
            \item \textbf{SuperH:}
                \begin{itemize}
                    \item Supaprastintas pertraukimų mechanizmas su ribotu prioritetų skaičiumi.
                    \item Mažesnė funkcionalumo įvairovė, pritaikyta įterptosioms sistemoms.
                \end{itemize}
        \end{itemize}
\end{itemize}

\section{Palaikomi duomenų tipai}
\subsection{PowerPC}
PowerPC architektūra aparatūros lygyje palaiko įvairius duomenų tipus, siekiant užtikrinti lankstumą ir našumą skirtingoms skaičiavimo užduotims.

\begin{itemize}
    \item \textbf{Sveikieji skaičiai:}
        \begin{itemize}
            \item Skaičiai koduojami naudojant \textbf{dvejeto papildomą kodą} (Two's Complement).
            \item Palaikomi 8, 16, 32 ir 64 bitų sveikieji skaičiai.
        \end{itemize}
    \item \textbf{Fiksuoto kablelio aritmetika:}
        \begin{itemize}
            \item Palaikoma sveikųjų skaičių fiksuoto kablelio aritmetika su registruose atliekamomis operacijomis.
        \end{itemize}
    \item \textbf{Slankiojo kablelio aritmetika:}
        \begin{itemize}
            \item Palaikoma aparatinė slankiojo kablelio aritmetika naudojant \textbf{IEEE 754} standartą.
            \item Palaikomi vieno tikslumo (32 bitų) ir dvigubo tikslumo (64 bitų) slankiojo kablelio skaičiai.
            \item Operacijos atliekamos naudojant specializuotus slankiojo kablelio registrus.
        \end{itemize}
    \item \textbf{Vektoriniai duomenų tipai:}
        \begin{itemize}
            \item Vėlesnėse PowerPC versijose palaikoma \textbf{Altivec (Velocity Engine)} technologija, skirta vektorinei aritmetikai.
            \item Leidžia vykdyti operacijas su vektoriais (pvz., 128 bitų registruose).
        \end{itemize}
    \item \textbf{Egzotiški duomenų tipai:}
        \begin{itemize}
            \item Specialūs duomenų tipai palaikomi naudojant vektorinius ir slankiojo kablelio registrus, tačiau aparatinis palaikymas dešimtainiams ar kompleksiniams skaičiams yra ribotas.
        \end{itemize}
\end{itemize}

\subsection{SuperH}
SuperH architektūra yra labiau supaprastinta ir pritaikyta įterptosioms sistemoms, todėl palaiko mažesnį duomenų tipų rinkinį.

\begin{itemize}
    \item \textbf{Sveikieji skaičiai:}
        \begin{itemize}
            \item Skaičiai koduojami naudojant \textbf{dvejeto papildomą kodą} (Two's Complement).
            \item Palaikomi 8, 16 ir 32 bitų sveikieji skaičiai.
        \end{itemize}
    \item \textbf{Fiksuoto kablelio aritmetika:}
        \begin{itemize}
            \item Palaikoma pagrindinė fiksuoto kablelio aritmetika naudojant bendrosios paskirties registrus.
        \end{itemize}
    \item \textbf{Slankiojo kablelio aritmetika:}
        \begin{itemize}
            \item Tik kai kuriuose SuperH modeliuose (pvz., SH-4) yra palaikomas slankiojo kablelio skaičiavimas pagal \textbf{IEEE 754} standartą.
            \item Atliekamos operacijos su vieno tikslumo (32 bitų) slankiojo kablelio skaičiais.
        \end{itemize}
    \item \textbf{Egzotiški duomenų tipai:}
        \begin{itemize}
            \item SuperH architektūra nesiūlo aparatinio palaikymo dešimtainiams ar kompleksiniams skaičiams.
        \end{itemize}
\end{itemize}

\section*{Duomenų tipų palyginimas}
\begin{itemize}
    \item \textbf{Panašumai:}
        \begin{itemize}
            \item Abi architektūros palaiko sveikuosius skaičius, koduojamus \textbf{dvejeto papildomu kodu}.
            \item Abi palaiko fiksuoto kablelio aritmetiką.
        \end{itemize}
    \item \textbf{Skirtumai:}
        \begin{itemize}
            \item \textbf{PowerPC:}
                \begin{itemize}
                    \item Palaiko vieno ir dvigubo tikslumo \textbf{IEEE 754} slankiojo kablelio aritmetiką.
                    \item Vėlesnėse versijose palaikomi vektoriniai duomenų tipai (\textbf{Altivec}).
                \end{itemize}
            \item \textbf{SuperH:}
                \begin{itemize}
                    \item Ribotas slankiojo kablelio palaikymas (tik kai kuriuose modeliuose, pvz., SH-4).
                    \item Nėra vektorinių ar egzotiškų duomenų tipų palaikymo.
                \end{itemize}
        \end{itemize}
\end{itemize}

\section{Sistemų greitaveika}
\subsection{PowerPC}
PowerPC procesoriai pasižymėjo didesne greitaveika, orientuota į našumą, todėl jie buvo naudojami serveriuose, asmeniniuose kompiuteriuose ir žaidimų konsolėse.

\begin{itemize}
    \item \textbf{Taktinių generatorių dažniai:}
        \begin{itemize}
            \item Ankstyvieji modeliai veikė nuo \textbf{50 MHz} iki \textbf{200 MHz}.
            \item Vėlesnėse versijose dažniai siekė iki \textbf{1–2 GHz} (pvz., PowerPC G4, G5).
        \end{itemize}
    \item \textbf{Ciklų skaičius komandai vykdyti:}
        \begin{itemize}
            \item Vidutinis komandų vykdymo laikas: \textbf{1–3 ciklai} dėl RISC architektūros.
            \item Kompleksinės instrukcijos, kaip slankiojo kablelio operacijos, galėjo užtrukti iki \textbf{5–10 ciklų}.
        \end{itemize}
    \item \textbf{Vidutinė greitaveika:}
        \begin{itemize}
            \item Ankstyvieji modeliai: apie \textbf{100–200 MIPS} (milijonų instrukcijų per sekundę).
            \item Vėlesni modeliai: daugiau nei \textbf{1000 MIPS} ir slankiojo kablelio našumas iki \textbf{10 GFLOPS}.
        \end{itemize}
    \item \textbf{Kainos ir našumo santykis:}
        \begin{itemize}
            \item PowerPC procesoriai buvo brangesni dėl didesnio našumo ir kompleksinių funkcijų.
            \item Naudojami aukštos klasės sistemose, kur reikalingas didelis našumas.
        \end{itemize}
\end{itemize}

\subsection{SuperH}
SuperH procesoriai buvo optimizuoti mažos galios ir kainos įterptosioms sistemoms, todėl jų greitaveika buvo kuklesnė, bet pakankama daugumai įterptųjų taikymų.

\begin{itemize}
    \item \textbf{Taktinių generatorių dažniai:}
        \begin{itemize}
            \item Ankstyvieji modeliai veikė nuo \textbf{20 MHz} iki \textbf{100 MHz}.
            \item SH-4 modeliai pasiekė iki \textbf{200 MHz}.
        \end{itemize}
    \item \textbf{Ciklų skaičius komandai vykdyti:}
        \begin{itemize}
            \item Vidutinis komandų vykdymo laikas: \textbf{1–2 ciklai} dėl efektyvios RISC architektūros.
            \item Didesnės skaičiavimo instrukcijos galėjo užtrukti iki \textbf{3–5 ciklų}.
        \end{itemize}
    \item \textbf{Vidutinė greitaveika:}
        \begin{itemize}
            \item Ankstyvieji modeliai: apie \textbf{20–50 MIPS}.
            \item Vėlesni SH-4 modeliai: iki \textbf{200–300 MIPS}.
        \end{itemize}
    \item \textbf{Kainos ir našumo santykis:}
        \begin{itemize}
            \item SuperH buvo pigesni dėl paprastesnio dizaino ir mažesnio našumo.
            \item Pasižymėjo geru kainos ir našumo santykiu mažos galios ir įterptosioms sistemoms.
        \end{itemize}
\end{itemize}

\section*{Greitaveikos palyginimas}
\begin{itemize}
    \item \textbf{PowerPC:}
        \begin{itemize}
            \item Aukštesni taktinių dažnių rodikliai (1–2 GHz vėlesniuose modeliuose).
            \item Aukštesnė greitaveika (iki 1000+ MIPS ir 10 GFLOPS).
            \item Didelė kaina dėl našumo ir papildomų funkcijų.
        \end{itemize}
    \item \textbf{SuperH:}
        \begin{itemize}
            \item Žemesni dažniai (20–200 MHz).
            \item Kuklesnė greitaveika (iki 300 MIPS).
            \item Geresnis kainos ir našumo santykis įterptosioms sistemoms.
        \end{itemize}
\end{itemize}

\section{Spartinančios atminties naudojimas}
\subsection{PowerPC}
PowerPC architektūra naudojo spartinančiąją atmintį (\textbf{cache}) siekiant pagerinti našumą, sumažinant prieigos prie pagrindinės atminties delsą.

\begin{itemize}
    \item \textbf{L1 spartinančioji atmintis:}
        \begin{itemize}
            \item Pirmo lygio (L1) spartinančioji atmintis yra integruota į procesorių.
            \item Dydis: nuo \textbf{8 KB} iki \textbf{64 KB} instrukcijoms ir duomenims (atskirai arba bendram naudojimui).
        \end{itemize}
    \item \textbf{L2 spartinančioji atmintis:}
        \begin{itemize}
            \item Antrasis lygis (L2) gali būti integruotas arba atskirai prijungtas.
            \item Dydis: nuo \textbf{128 KB} iki \textbf{1 MB}.
        \end{itemize}
    \item \textbf{L3 spartinančioji atmintis:}
        \begin{itemize}
            \item Kai kuriuose PowerPC modeliuose (pvz., PowerPC G5) palaikoma trečio lygio (L3) spartinančioji atmintis.
            \item Dydis: iki \textbf{4 MB}.
        \end{itemize}
\end{itemize}

Spartinančioji atmintis leido PowerPC procesoriams efektyviai apdoroti duomenis, sumažinant prieigos laiką prie pagrindinės atminties, kas ypač svarbu aukštos spartos sistemose.

\subsection{SuperH}
SuperH architektūra taip pat naudojo spartinančiąją atmintį, tačiau jos dydis ir funkcionalumas buvo kuklesnis, atsižvelgiant į architektūros pritaikymą įterptosioms sistemoms.

\begin{itemize}
    \item \textbf{L1 spartinančioji atmintis:}
        \begin{itemize}
            \item Pirmo lygio (L1) spartinančioji atmintis yra integruota į procesorių.
            \item Dydis: nuo \textbf{4 KB} iki \textbf{16 KB} (bendra instrukcijų ir duomenų spartinančioji atmintis).
        \end{itemize}
    \item \textbf{L2 spartinančioji atmintis:}
        \begin{itemize}
            \item Spartinančioji atmintis L2 lygyje buvo neprivaloma ir tik kai kuriuose SH-4 modeliuose.
            \item Dydis: iki \textbf{128 KB}.
        \end{itemize}
\end{itemize}

Dėl mažesnio spartinančiosios atminties dydžio SuperH buvo mažiau našūs lyginant su PowerPC, tačiau atitiko įterptųjų sistemų poreikius dėl mažos galios suvartojimo.

\section*{Spartinančios atminties palyginimas}
\begin{itemize}
    \item \textbf{PowerPC:}
        \begin{itemize}
            \item Naudojamos L1, L2 ir kai kuriais atvejais L3 spartinančiosios atminties lygiai.
            \item L1 dydis: iki \textbf{64 KB}, L2 dydis: iki \textbf{1 MB}, L3 dydis: iki \textbf{4 MB}.
        \end{itemize}
    \item \textbf{SuperH:}
        \begin{itemize}
            \item Tik L1 spartinančioji atmintis yra standartinė.
            \item L1 dydis: iki \textbf{16 KB}, L2 dydis (ne visada): iki \textbf{128 KB}.
        \end{itemize}
\end{itemize}

\section{Tipinės taikymo sritys}
\subsection{PowerPC}
PowerPC architektūra buvo plačiai naudojama sistemose, kur reikėjo didelio našumo, patikimumo ir lankstumo. Ji buvo pritaikyta serveriuose, asmeniniuose kompiuteriuose, žaidimų konsolėse ir įterptosioms sistemoms. Dėl savo RISC dizaino ir pažangaus spartinančiosios atminties valdymo PowerPC sugebėjo atlikti sudėtingus skaičiavimus greitai ir efektyviai.

\textbf{Konkretus pavyzdys:} PowerPC procesoriai buvo naudojami žaidimų konsolėje \textbf{Xbox 360}. Konsolėje veikė trijų branduolių PowerPC procesorius, veikiantis \textbf{3.2 GHz} dažniu. Šis procesorius leido sklandžiai vykdyti aukštos kokybės žaidimus, atlikdamas sudėtingus grafinius ir fizikos skaičiavimus.

\subsection{SuperH}
SuperH architektūra buvo sukurta kaip energiją taupanti ir kompaktiška sistema, skirta įterptosioms sistemoms, mobiliems įrenginiams ir buitinei elektronikai. Ji buvo naudojama tokiose srityse kaip automobilių pramonė, pramoniniai valdikliai, mobilieji telefonai ir multimedijos įrenginiai, kur reikėjo mažų sąnaudų ir patikimo veikimo.

\textbf{Konkretus pavyzdys:} SuperH procesoriai buvo plačiai naudojami \textbf{SEGA Dreamcast} žaidimų konsolėje. Konsolėje buvo įdiegtas \textbf{SH-4} procesorius, veikiantis \textbf{200 MHz} dažniu. Procesorius palaikė slankiojo kablelio skaičiavimus ir buvo naudojamas tiek žaidimų vykdymui, tiek grafinių vaizdų apdorojimui realiu laiku.

\section*{Taikymo sričių palyginimas}
\begin{itemize}
    \item \textbf{PowerPC:} Naudota aukšto našumo sistemose – serveriuose, asmeniniuose kompiuteriuose (pvz., Apple Mac) ir žaidimų konsolėse (pvz., Xbox 360).
    \item \textbf{SuperH:} Naudota energiją taupančiose įterptosiose sistemose, buitinėje elektronikoje ir žaidimų konsolėse (pvz., SEGA Dreamcast).
\end{itemize}

\section{Programinė įranga ir įrankiai}
\subsection{PowerPC}
PowerPC architektūrai buvo sukurta daug programinės įrangos, nes ji buvo naudojama įvairiose srityse: asmeniniuose kompiuteriuose, serveriuose ir žaidimų konsolėse.

\begin{itemize}
    \item \textbf{Programinės įrangos kiekis ir prieinamumas:}
        \begin{itemize}
            \item Daugybė programų buvo sukurtos \textbf{Apple Macintosh} kompiuteriams su PowerPC procesoriais (pvz., Mac OS 7–9, Mac OS X iki 10.5 versijos).
            \item Naudota žaidimų konsolėse, tokiose kaip \textbf{Xbox 360}, \textbf{Nintendo Wii}, kur buvo pritaikyti žaidimai ir sisteminė programinė įranga.
            \item Šiandien kai kurios programos vis dar prieinamos per emuliatorius ir senų sistemų palaikymo bendruomenes.
        \end{itemize}
    \item \textbf{Prieinami kompiliatoriai ir įrankiai:}
        \begin{itemize}
            \item \textbf{GCC (GNU Compiler Collection):} Palaiko PowerPC platformą ir naudojamas kompiliavimui.
            \item \textbf{IBM XL C/C++:} Aukšto našumo kompiliatorius, skirtas PowerPC serveriams.
            \item \textbf{LLVM/Clang:} Naudotas kai kurioms PowerPC implementacijoms.
            \item Derintojai: \textbf{GDB (GNU Debugger)}, \textbf{IBM Debugger}.
            \item Surinkėjai: GNU Assembler (GAS), IBM Assembler.
        \end{itemize}
    \item \textbf{Programinės įrangos bibliotekos:}
        \begin{itemize}
            \item \textbf{libc}: Standartinė C biblioteka.
            \item \textbf{AltiVec bibliotekos:} Naudojamos vektorinių skaičiavimų optimizacijai.
            \item \textbf{POSIX bibliotekos}: Sistemos funkcijoms ir multithreading'ui.
        \end{itemize}
\end{itemize}

\subsection{SuperH}
SuperH architektūrai buvo sukurta mažiau programinės įrangos dėl jos orientacijos į įterptąsias sistemas ir buitinę elektroniką.

\begin{itemize}
    \item \textbf{Programinės įrangos kiekis ir prieinamumas:}
        \begin{itemize}
            \item SuperH procesoriai naudoti \textbf{SEGA Dreamcast} konsolėje, kur buvo sukurta daugybė žaidimų ir multimedijos programų.
            \item Plačiai naudoti \textbf{įterptųjų sistemų} srityje (automobilių valdikliai, pramoniniai valdikliai).
            \item Šiandien SuperH programinė įranga yra prieinama per emuliatorius (pvz., \textbf{Shenmue} emuliatoriai) ir atviro kodo bendruomenes.
        \end{itemize}
    \item \textbf{Prieinami kompiliatoriai ir įrankiai:}
        \begin{itemize}
            \item \textbf{GCC (GNU Compiler Collection):} Palaiko SuperH platformą.
            \item \textbf{Renesas kompiliatoriai:} Oficialūs kompiliatoriai, skirti SH-4 ir kitoms versijoms.
            \item Derintojai: \textbf{GDB}, Renesas derintojai.
            \item Surinkėjai: GNU Assembler (GAS).
            \item Profiliuotojai: \textbf{gprof} ir įrankiai iš Renesas.
        \end{itemize}
    \item \textbf{Programinės įrangos bibliotekos:}
        \begin{itemize}
            \item \textbf{newlib}: Lengva C biblioteka, skirta įterptosioms sistemoms.
            \item \textbf{libm}: Matematinė biblioteka fiksuoto ir slankiojo kablelio operacijoms.
            \item Specializuotos bibliotekos, skirtos valdyti \textbf{serijines sąsajas} ir periferinius įrenginius.
        \end{itemize}
\end{itemize}

\section*{Programinės įrangos palyginimas}
\begin{itemize}
    \item \textbf{PowerPC:}
        \begin{itemize}
            \item Naudota asmeniniuose kompiuteriuose, serveriuose ir žaidimų konsolėse.
            \item Turtingas įrankių ir bibliotekų rinkinys, palaikomas GNU ir IBM ekosistemų.
        \end{itemize}
    \item \textbf{SuperH:}
        \begin{itemize}
            \item Orientuota į įterptąsias sistemas ir žaidimų konsoles.
            \item Supaprastintas įrankių rinkinys su specializuotomis bibliotekomis mažoms sistemoms.
        \end{itemize}
\end{itemize}

\section{Emuliatorių prieinamumas}
\subsection{PowerPC}
PowerPC architektūrai yra prieinami keli emuliatoriai, leidžiantys naudoti seną programinę įrangą ir žaidimus moderniuose kompiuteriuose:

\begin{itemize}
    \item \textbf{QEMU:}
        \begin{itemize}
            \item Atvirojo kodo emuliatorius, palaikantis PowerPC procesorių emuliaciją.
            \item URL: \url{https://www.qemu.org/}
        \end{itemize}
    \item \textbf{SheepShaver:}
        \begin{itemize}
            \item Emuliatorius, skirtas Mac OS (7.5.2–9.0.4) su PowerPC procesoriais.
            \item URL: \url{https://sheepshaver.cebix.net/}
        \end{itemize}
    \item \textbf{PearPC:}
        \begin{itemize}
            \item PowerPC emuliatorius, leidžiantis paleisti Mac OS X (10.1–10.4) ant x86 architektūros.
            \item URL: \url{https://pearpc.sourceforge.net/}
        \end{itemize}
\end{itemize}

\subsection{SuperH}
SuperH architektūrai taip pat yra prieinami emuliatoriai, dažniausiai naudojami žaidimų ir įterptųjų sistemų programinei įrangai emuliuoti:

\begin{itemize}
    \item \textbf{Demul:}
        \begin{itemize}
            \item Emuliatorius, skirtas \textbf{SEGA Dreamcast} ir SH-4 architektūros žaidimams.
            \item URL: \url{https://demul.dev/}
        \end{itemize}
    \item \textbf{GXemul:}
        \begin{itemize}
            \item Atvirojo kodo emuliatorius, palaikantis SH-4 ir kitų SuperH procesorių emuliaciją.
            \item URL: \url{http://gxemul.sourceforge.net/}
        \end{itemize}
    \item \textbf{MAME (Multiple Arcade Machine Emulator):}
        \begin{itemize}
            \item Palaiko įvairias SuperH platformas, įskaitant arkadinius žaidimus ir SEGA sistemas.
            \item URL: \url{https://www.mamedev.org/}
        \end{itemize}
\end{itemize}

\section*{Emuliatorių palyginimas}
\begin{itemize}
    \item \textbf{PowerPC:} Emuliatoriai, tokie kaip QEMU ir SheepShaver, palaiko Mac OS ir kitas sistemas.
    \item \textbf{SuperH:} Emuliatoriai, tokie kaip Demul ir GXemul, skirti žaidimų konsolėms ir įterptosioms sistemoms.
\end{itemize}

\bibliographystyle{plain}
\bibliography{literatura} 

\end{document}
