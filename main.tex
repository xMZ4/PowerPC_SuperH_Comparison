\documentclass{article}
\usepackage{graphicx} % Paveikslėliams įterpti
\usepackage{geometry} % Paraščių keitimui
\geometry{a4paper, margin=1in}
\usepackage{hyperref} % Nuorodoms dokumente
\usepackage{float}    % Paveikslų ir lentelių pozicionavimui

\title{PowerPC ir SuperH palyginimas}
\author{Mateušas Zlosnikas}
\date{2024 m. gruodis}

\begin{document}

\maketitle

\tableofcontents % Turinys

\section{Įvadas}
Šiame dokumente palyginamos dvi procesorių architektūros: \textbf{PowerPC} ir \textbf{SuperH}. Abi architektūros turėjo reikšmingą įtaką skirtingoms kompiuterių sritims – nuo įterptųjų sistemų iki aukštos spartos skaičiavimų.

\section{Elementinė kompiuterių bazė}
\subsection{PowerPC}
PowerPC procesoriai buvo sukurti naudojant \textbf{VLSI} (labai didelės integracijos masto) technologiją ir palaipsniui evoliucionavo iki šiuolaikinių monokristalinių mikroprocesorių. Ankstyvieji modeliai buvo paremti \textbf{RISC} (sumažinto komandų rinkinio kompiuterio) dizainu, siekiant aukšto našumo ir mažo sudėtingumo.

\subsection{SuperH}
SuperH procesoriai taip pat buvo pagaminti naudojant \textbf{VLSI} technologiją. Jų dizainas buvo optimizuotas mažam energijos suvartojimui ir kompaktiškam dydžiui, todėl jie plačiai naudoti įterptosiose sistemose, automobilių pramonėje ir mobiliojoje elektronikoje.

\section{Fizinės įrangos savybės}
\subsection{PowerPC}
\begin{itemize}
    \item \textbf{Dydis:} Ankstyvieji modeliai buvo didesni, tačiau technologinė pažanga leido procesoriams sumažėti.
    \item \textbf{Energijos suvartojimas:} Nuo vidutinio iki didelio (50–100 W).
\end{itemize}

\subsection{SuperH}
\begin{itemize}
    \item \textbf{Dydis:} Kompaktiškas dizainas, užimantis vos kelis kvadratinius milimetrus.
    \item \textbf{Energijos suvartojimas:} Itin mažas – nuo 0,5 W iki 10 W.
\end{itemize}

\section{Architektūrų tipai}
\subsection{PowerPC}
PowerPC procesoriai naudoja \textbf{RISC} architektūrą, kuri yra \textbf{registrinio tipo}. Tai reiškia, kad dauguma operacijų atliekamos tarp registrų, o ne tiesiogiai naudojant atmintį. Toks sprendimas leidžia sumažinti delsą ir padidinti našumą.

\begin{itemize}
    \item \textbf{Architektūros tipas:} Registrinė, RISC.
    \item \textbf{Savybės:}
        \begin{itemize}
            \item Fiksuoto ilgio instrukcijos.
            \item Daugybė bendros paskirties registrų.
        \end{itemize}
\end{itemize}

\subsection{SuperH}
SuperH procesoriai taip pat naudoja \textbf{registrinio tipo RISC} architektūrą. Šis dizainas leidžia procesoriams veikti efektyviai tiek energijos, tiek našumo atžvilgiu, todėl jie ypač tinkami įterptosioms sistemoms.

\begin{itemize}
    \item \textbf{Architektūros tipas:} Registrinė, RISC.
    \item \textbf{Savybės:}
        \begin{itemize}
            \item Mažas ir efektyvus komandų rinkinys.
            \item Optimizuotas energijos taupymui.
        \end{itemize}
\end{itemize}
\section{Adresavimo tipai}
\subsection{PowerPC}
PowerPC procesoriai naudoja \textbf{trijų adresų} architektūrą. Instrukcijos dažniausiai turi du šaltinio operandus ir vieną rezultatą, kuris įrašomas į registrą. Ši konstrukcija leidžia atlikti daugiau operacijų per mažesnį instrukcijų skaičių, taip pagerinant našumą.

\begin{itemize}
    \item \textbf{Adresų skaičius:} Trijų adresų.
    \item \textbf{Pavyzdys:} \( R_d = R_s + R_t \), kur \( R_d \) – rezultato registras, o \( R_s \) ir \( R_t \) – šaltinio registrai.
\end{itemize}

\subsection{SuperH}
SuperH procesoriai naudoja \textbf{dviejų adresų} architektūrą. Instrukcija naudoja vieną šaltinio operandą ir rezultatą, kuris gali būti įrašomas į tą patį registrą arba atmintį. Tai supaprastina procesoriaus instrukcijų dekodavimą, tačiau kartais reikalauja papildomų instrukcijų norint atlikti kompleksines operacijas.

\begin{itemize}
    \item \textbf{Adresų skaičius:} Dviejų adresų.
    \item \textbf{Pavyzdys:} \( R_d = R_d + R_s \), kur \( R_d \) veikia kaip tiek šaltinis, tiek rezultatas.
\end{itemize}

\section{Registrai}
\subsection{PowerPC}
PowerPC architektūra turi tiek \textbf{bendrosios paskirties}, tiek \textbf{specializuotus registrus}. Šie registrai yra svarbūs vykdant RISC tipo instrukcijas ir optimizuojant skaičiavimus.

\begin{itemize}
    \item \textbf{Bendrosios paskirties registrai:} 
        \begin{itemize}
            \item PowerPC turi \textbf{32 bendrosios paskirties registrus} (GPR – General Purpose Registers).
            \item Registrų \textbf{duomenų plotis}: 32 bitai ankstyvuose modeliuose, vėlesniuose modeliuose – 64 bitai.
        \end{itemize}
    \item \textbf{Specializuoti registrai:}
        \begin{itemize}
            \item \textbf{LR (Link Register):} Saugoti grįžimo adresą funkcijų kvietimuose.
            \item \textbf{CTR (Count Register):} Naudojamas ciklams ir skaičiavimams.
            \item \textbf{FPSCR (Floating Point Status and Control Register):} Valdo slankaus kablelio operacijas.
            \item \textbf{Specialūs vektoriniai registrai:} Naudojami multimedijos ir signalų apdorojimui (vėlesniuose modeliuose).
        \end{itemize}
\end{itemize}

\subsection{SuperH}
SuperH architektūra turi mažesnį skaičių registrų, tačiau jie yra efektyviai panaudojami, kadangi SuperH yra skirta įterptosioms sistemoms, kur mažas energijos suvartojimas ir greitis yra itin svarbūs.

\begin{itemize}
    \item \textbf{Bendrosios paskirties registrai:}
        \begin{itemize}
            \item SuperH turi \textbf{16 bendrosios paskirties registrų} (R0–R15).
            \item Registrų \textbf{duomenų plotis}: 32 bitai.
        \end{itemize}
    \item \textbf{Specializuoti registrai:}
        \begin{itemize}
            \item \textbf{PC (Program Counter):} Laiko vykdomos instrukcijos adresą.
            \item \textbf{SR (Status Register):} Saugomi procesoriaus būsenos bitai.
            \item \textbf{SP (Stack Pointer):} Valdo steko operacijas.
            \item \textbf{MAC (Multiply-Accumulate Register):} Naudojamas dauginimo ir sumavimo operacijoms.
        \end{itemize}
\end{itemize}

\section*{Registrų palyginimas}
\begin{itemize}
    \item \textbf{PowerPC:} 32 bendrosios paskirties registrai, duomenų plotis iki 64 bitų, daug specializuotų registrų.
    \item \textbf{SuperH:} 16 bendrosios paskirties registrų, 32 bitų duomenų plotis, mažiau specializuotų registrų.
\end{itemize}

\section{Požymių bitai}
\subsection{PowerPC}
PowerPC architektūroje požymių bitai yra naudojami tam tikriems procesoriaus būsenos signalams valdyti. Jie yra saugomi specializuotame registruose, tokiuose kaip \textbf{Condition Register (CR)} ir \textbf{Floating Point Status and Control Register (FPSCR)}.

\begin{itemize}
    \item \textbf{CR (Condition Register):}
        \begin{itemize}
            \item CR registras turi \textbf{8 laukus} (CR0–CR7), kur kiekvienas laukas yra 4 bitų.
            \item Požymių bitai atspindi instrukcijų rezultatus, tokius kaip:
                \begin{itemize}
                    \item \textbf{EQ (Equal):} Reikšmės lygios.
                    \item \textbf{LT (Less Than):} Reikšmė mažesnė.
                    \item \textbf{GT (Greater Than):} Reikšmė didesnė.
                    \item \textbf{SO (Summary Overflow):} Viršijo aritmetinės operacijos reikšmę.
                \end{itemize}
        \end{itemize}
    \item \textbf{FPSCR (Floating Point Status and Control Register):}
        \begin{itemize}
            \item Naudojamas slankaus kablelio operacijų požymių bitams saugoti.
            \item Valdo tokias būsenas kaip **viršpildymas**, **nulio dalyba** ir **neapibrėžtos reikšmės**.
        \end{itemize}
\end{itemize}

\subsection{SuperH}
SuperH architektūroje požymių bitai taip pat egzistuoja, tačiau jų kiekis ir funkcionalumas yra supaprastintas, atsižvelgiant į architektūros efektyvumą.

\begin{itemize}
    \item \textbf{SR (Status Register):}
        \begin{itemize}
            \item SR registras saugo požymių bitus, kurie nurodo procesoriaus būseną.
            \item Dažniausiai naudojami bitai:
                \begin{itemize}
                    \item \textbf{T (T-bit):} Rezultato tikrinimo bitas (naudojamas sąlyginiams šakoms).
                    \item \textbf{I (Interrupt Mask):} Pertraukčių valdymas.
                    \item \textbf{S (Supervisor Mode):} Indikuoja veikimą administratoriaus režimu.
                \end{itemize}
        \end{itemize}
\end{itemize}

\section*{Požymių bitų palyginimas}
\begin{itemize}
    \item \textbf{PowerPC:} Išsamesnė požymių sistema su **CR** ir **FPSCR** registrais, palaikančiais aritmetinių ir slankaus kablelio operacijų būsenas.
    \item \textbf{SuperH:} Paprastesnė sistema su **SR** registru, kuriame yra pagrindiniai bitai, reikalingi šakoms ir pertrauktims valdyti.
\end{itemize}


\bibliographystyle{plain}
\bibliography{literatura} 

\end{document}
